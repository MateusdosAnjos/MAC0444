\documentclass[12pt]{article}

\usepackage[utf8]{inputenc}
%\usepackage[portuguese]{babel}
\usepackage{float}

\title{MAC0444 - Sistemas Baseados em Conhecimento \\
Lista de Exercícios No. 1
}
\author{Mateus Agostinho dos Anjos\\NUSP 9298191}
\date{\today}

\begin{document}
	\maketitle
	\begin{itemize}
		\item[1 -]
			$D$ = Domínio\\
			$I$ = Interpretação (apenas os Verdadeiros)\\
			\begin{itemize}
				\item[a)]
					\hfill\\
					Queremos: a)F b)V c)V\\
					Portanto, definimos:\\
					$D = \lbrace 0, 1, 2, 3 \rbrace$\\
					$I = \lbrace P(1, 2), P(2, 3), P(3, 1) \rbrace$\\ \\
					Para a)F\\
					$\forall x \forall y \forall z ((P(x,y) \wedge P(y,z)) \rightarrow P(x, z))$\\
					É falso para $x/1 \ , \ y/2, \ z/3$\\
					\begin{center}	
						$((P(1,2) \wedge P(2,3)) \rightarrow P(1, 3))$\\
						$((V \wedge V) \rightarrow F)$\\
						$F$
					\end{center}
					Para b)V\\
					$\forall x \forall y ((P(x,y) \wedge P(y,x)) \rightarrow x = y)$\\
					Não temos interpretações que satisfaçam $((P(x,y) \wedge P(y,x))$, portanto
					a implicação será sempre verdadeira, uma vez que o lado esquerdo dela é
					sempre falso. ($F \rightarrow F$ é $V$ e $F \rightarrow V$ é $V$)\\
					
					Para c)V\\
					$\forall x \forall y (P(a,y) \rightarrow P(x, b))$\\
					Fixando $a = 0$ temos que $P(a, y) = P(0, y)$\\
					Perceba que $\forall y P(0, y)$ é sempre falso, fazendo a implicação c)
					sempre verdadeira. (caso análogo ao item b) acima)\\
					Basta, então escolhermos $b = 2$ (ou qualquer outro elemento do domínio)
					
					\item[b)]
						\hfill\\
						Queremos: a)V b)F c)V\\
						Portanto, definimos:\\
						$D = \lbrace 1, 2 \rbrace$\\
						$I = \lbrace P(1, 1), P(1, 2), P(2, 1), P(2, 2)\rbrace$\\ \\
						Para a)V\\
						$\forall x \forall y \forall z ((P(x,y) \wedge P(y,z)) \rightarrow P(x, z))$\\
						Trivial a verificação.\\
						(Testar toda a tripla $x, y, z = (1, 1, 1), (1, 1, 2),
						(1, 2, 1), (1, 2, 2), \ \dots$)\\
						
						Para b)F\\
						$\forall x \forall y ((P(x,y) \wedge P(y,x)) \rightarrow x = y)$\\
						Temos $x/1$ e $y/2$\\
						\begin{center}
							$(P(1,2) \wedge P(2,1)) \rightarrow 1 = 2$\\
							$V \wedge V \rightarrow F$\\
							$F$
						\end{center}
						Para c)V\\
						$\forall x \forall y (P(a,y) \rightarrow P(x, b))$\\
						Basta escolhermos $a = 2$ e $b = 1$ e ficamos com:\\
						$\forall x \forall y (P(2,y) \rightarrow P(x, 1))$\\
						Restando as opções:\\
						$(P(2,1) \rightarrow P(1, 1))$  ($(V \rightarrow V)$ é $V$)\\
						$(P(2,1) \rightarrow P(2, 1))$  ($(V \rightarrow V)$ é $V$)\\
						$(P(2,2) \rightarrow P(1, 1))$  ($(V \rightarrow V)$ é $V$)\\
						$(P(2,2) \rightarrow P(2, 2))$  ($(V \rightarrow V)$ é $V$)\\
						Mostrando que c) é sempre Verdadeiro como queríamos.
					
					
					
			\end{itemize}
		
	\end{itemize}
\end{document}