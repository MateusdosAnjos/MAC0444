\documentclass[12pt]{article}

\usepackage[utf8]{inputenc}
%\usepackage[portuguese]{babel}
\usepackage{float}

\title{MAC0444 - Sistemas Baseados em Conhecimento \\
Lista de Exercícios No. 1
}
\author{Mateus Agostinho dos Anjos\\NUSP 9298191}
\date{\today}

\begin{document}
	\maketitle
	\begin{itemize}
		\item[\textbf{1 -}]
			\hfill\newline
			$D$ = Domínio\\
			$I$ = Interpretação (apenas os Verdadeiros)\\
			\begin{itemize}
				\item[\textbf{A)}]
					\hfill\\
					\textbf{Queremos: a)F b)V c)V}\\
					Portanto, definimos:\\
					$D = \lbrace 0, 1, 2, 3 \rbrace$\\
					$I = \lbrace P(1, 2), P(2, 3), P(3, 1) \rbrace$\\ \\
					\textbf{Para a)F}\\
					$\forall x \forall y \forall z ((P(x,y) \wedge P(y,z)) \rightarrow P(x, z))$\\
					É falso para $x/1 \ , \ y/2, \ z/3$\\
					\begin{center}	
						$((P(1,2) \wedge P(2,3)) \rightarrow P(1, 3))$\\
						$((V \wedge V) \rightarrow F)$\\
						$F$
					\end{center}
					\textbf{Para b)V}\\
					$\forall x \forall y ((P(x,y) \wedge P(y,x)) \rightarrow x = y)$\\
					Não temos interpretações que satisfaçam $((P(x,y) \wedge P(y,x))$, portanto
					a implicação será sempre verdadeira, uma vez que o lado esquerdo dela é
					sempre falso. ($F \rightarrow F$ é $V$ e $F \rightarrow V$ é $V$)\\
					
					\textbf{Para c)V}\\
					$\forall x \forall y (P(a,y) \rightarrow P(x, b))$\\
					Fixando $a = 0$ temos que $P(a, y) = P(0, y)$\\
					Perceba que $\forall y P(0, y)$ é sempre falso, fazendo a implicação c)
					sempre verdadeira. (caso análogo ao item b) acima)\\
					Basta, então escolhermos $b = 2$ (ou qualquer outro elemento do domínio)
					
					\item[\textbf{B)}]
						\hfill\\
						\textbf{Queremos: a)V b)F c)V}\\
						Portanto, definimos:\\
						$D = \lbrace 1, 2 \rbrace$\\
						$I = \lbrace P(1, 1), P(1, 2), P(2, 1), P(2, 2)\rbrace$\\ \\
						\textbf{Para a)V}\\
						$\forall x \forall y \forall z ((P(x,y) \wedge P(y,z)) \rightarrow P(x, z))$\\
						Testar:\\
						$x, y, z = (1, 1, 1) , (1, 1, 2), (1, 2, 1), (1, 2, 2), (2, 1, 1),
						(2, 1, 2), (2, 2, 1), (2, 2, 2)$\\
						
						\textbf{Para b)F}\\
						$\forall x \forall y ((P(x,y) \wedge P(y,x)) \rightarrow x = y)$\\
						Temos $x/1$ e $y/2$\\
						\begin{center}
							$(P(1,2) \wedge P(2,1)) \rightarrow 1 = 2$\\
							$V \wedge V \rightarrow F$\\
							$F$
						\end{center}
						\textbf{Para c)V}\\
						$\forall x \forall y (P(a,y) \rightarrow P(x, b))$\\
						Basta escolhermos $a = 2$ e $b = 1$ e ficamos com:\\
						$\forall x \forall y (P(2,y) \rightarrow P(x, 1))$\\
						Restando as opções:\\
						$P(2,1) \rightarrow P(1, 1)$\\
						$P(2,1) \rightarrow P(2, 1)$\\
						$P(2,2) \rightarrow P(1, 1)$ \\
						$P(2,2) \rightarrow P(2, 2)$\\
						Mostrando que c) é sempre Verdadeiro como queríamos.
						
						
					\item[\textbf{C)}]
						\hfill\\
						\textbf{Queremos: a)V b)V c)F}\\
						Portanto, definimos:\\
						$D = \lbrace 1, 2 \rbrace$\\
						$I = \lbrace P(1, 1), P(2, 2) \rbrace$\\ \\
						\textbf{Para a)V}\\
						$\forall x \forall y \forall z ((P(x,y) \wedge P(y,z)) \rightarrow P(x, z))$\\
						Testar:\\
						$x, y, z = (1, 1, 1) , (1, 1, 2), (1, 2, 1), (1, 2, 2), (2, 1, 1),
						(2, 1, 2), (2, 2, 1), (2, 2, 2)$\\
						\\
						\textbf{Para b)V}\\
						$\forall x \forall y ((P(x,y) \wedge P(y,x)) \rightarrow x = y)$\\
						Veja que para todo par $x \neq y$ o lado esquerdo será falso, fazendo
						a implicação ser verdadeira. Já para $x = y$ a implicação será verdadeira,
						pois $P(1, 1)$ e $P(2, 2)$ são verdadeiros.\\
						\\
						\textbf{Para c)F}\\
						$\forall x \forall y (P(a,y) \rightarrow P(x, b))$\\
						Basta escolher $a = 1$ e $b = 2$\\
						Teremos: $\forall x \forall y (P(1,y) \rightarrow P(x, 2))$\\
						Tendo: $(P(1,1) \rightarrow P(1, 2))$ como exemplo de implicação falsa. 
			\end{itemize}
		\newpage
		\item[\textbf{2 -}]
			\hfill\newline
			Definição de predicados:\\ \\
			$MembroClubeAlpino(x)$ = x é membro do Clube Alpino\\
			$Esquiador(x)$ = x é esquiador\\
			$Alpinista(x)$ = x é alpinista\\
			$GostaDe(x, y)$ = x gosta de y\\
			\begin{itemize}
				\item[\textbf{A) }]
					\hfill\newline
					\begin{center}
						Base de Conhecimento (KB):\\ \hfill \\
						\begin{tabular}{l|l}
							1. & $MembroClubeAlpino(Tony)$\\
							2. & $MembroClubeAlpino(Mike)$\\
							3. & $MembroClubeAlpino(John)$\\
							4. & $\forall x ((MembroClubeAlpino(x) \ \wedge \ \neg Esquiador(x)) 
							\rightarrow Alpinista(x)) $ \\
							5. & $\forall x (Alpinista(x) \rightarrow \neg GostaDe(x, chuva)) $\\
							6. & $\forall x (\neg GostaDe(x, neve) \rightarrow \neg Esquiador(x))$\\
							7. & $\forall x (GostaDe(Tony, x) \rightarrow \neg GostaDe(Mike, x))$\\
							8. & $\forall x (\neg GostaDe(Tony, x) \rightarrow GostaDe(Mike, x))$\\
							9. & $GostaDe(Tony, chuva)$\\
							10. & $GostaDe(Tony, neve)$\\			
						\end{tabular}
					\end{center}
				\item[\textbf{B) }]
					\hfill\newline
					Sabemos que Tony gosta de chuva e de neve (KB: 9 e 10), portanto
					Mike não gosta de chuva nem de neve (KB: 7). \textbf{Como Mike não gosta
					de neve ele não é esquiador (KB: 6)} e a partir disso podemos
					concluir que \textbf{Mike é alpinista, já que é membro do clube alpino e
					não é esquiador (KB: 4)}.
				\item[\textbf{C) }]
					\hfill\newline	
					Retirando (KB: 7)  $\forall x (GostaDe(Tony, x) \rightarrow \neg GostaDe(Mike, 
					x))$ temos pelo menos uma interpretação em que Mike gosta de neve
					e é esquiador, portanto a prova acima já não é mais válida.\\
					Além disso \textbf{não é possível determinar se Mike ou John não gostam de
					neve para podermos concluir, a partir de (KB: 6), que um deles não é
					esquiador}, portanto temos interpretações em que eles são esquiadores,
					logo não podemos afirmar que existe um membro do clube alpino que é alpinista mas 
					não é esquiador.\\
					Veja o contra-exemplo abaixo que mostra como satisfazer toda a base de 
					conhecimento e não ter um membro do clube alpino que é alpinista e não
					é esquiador:\\ \\
					\begin{center}
						Tony gosta de chuva e neve, não é alpinista e é esquiador\\
						John não gosta de chuva e  gosta de neve, é alpinista e esquiador\\
						Mike não gosta de chuva e gosta de neve, é alpinista e esquiador\\
					\end{center}
					
					
			\end{itemize}
	\end{itemize}
\end{document}