\documentclass[12pt]{article}

\usepackage[utf8]{inputenc}
\usepackage[portuguese]{babel}
\usepackage{float}
\usepackage{graphicx}
\usepackage{amssymb}

\title{MAC0444 - Sistemas Baseados em Conhecimento \\
Lista de Exercícios No. 3
}
\author{Mateus Agostinho dos Anjos\\NUSP 9298191}
\date{\today}

\begin{document}
	\maketitle
	\begin{itemize}
		\item[\textbf{1 -}]
			\hfill\newline
			\begin{itemize}
				\item[\textbf{a) }]
					\hfill\newline
					O conceito abaixo significa:\\
					Um humano que não é do sexo feminino, casado(a) com um médico(a) e que tem filho(a) que
					é médico(a) ou professor(a). 
				\item[\textbf{b) }]
					\hfill\newline
					Primeiro devemos fazer algumas considerações:\\
					(Tais considerações são consideradas como senso comum, porém
					precisam ser especificadas)
					\subitem\textbf{•}
					Todos as 4 pessoas citadas são humanas e diferentes entre si.
					\subitem\textbf{•}
					Jõao e Pedro não são do sexo feminino enquanto Marta e Olívia são do sexo feminino.
					\subitem\textbf{•}
					Geriatra e cardiologista são profissões que caracterizam um médico(a).
					\newline
					A partir disso podemos dizer que \textbf{Pedro pertence a esse conceito}, pois é humano,
					não é do sexo feminino, é casado com Olívia que é uma médica e tem o filho João que é
					médico.
					\newpage
				\item[\textbf{c) }]
					\hfill\newline
					A resposta depende da pessoa excluída:
					\subitem\textbf{•}
					Se excluirmos João, não podemos concluir que Pedro tem um filho(a) que é médico(a) ou
					professor(a), portanto Pedro já não é uma resposta e não podemos concluir que
					alguma outra pessoa se enquadra no conceito citado, portanto ao excluir João não
					podemos afirmar que algum deles pertence a esse conceito.
					\subitem\textbf{•}
					Se excluirmos Marta, a resposta do item b) não se altera, pois ela não tem
					nenhuma influência na questão.
					\subitem\textbf{•}
					Se excluirmos Pedro, não podemos responder que ele se enquadra no conceito, uma
					vez que ele nem faz parte do domínio, além disso não podemos concluir que
					alguma outra pessoa se enquadra no conceito citado, portanto ao excluir Pedro não
					podemos afirmar que algum deles pertence a esse conceito.
					\subitem\textbf{•}
					Se excluirmos Olívia, não podemos concluir que Pedro é casado, portanto Pedro já não 
					é uma resposta e não podemos concluir que alguma outra pessoa se enquadra no conceito
					citado, portanto ao excluir Olívia não
					podemos afirmar que algum deles pertence a esse conceito.
			\end{itemize}
		\item[\textbf{2 -}]
		\item[\textbf{3 -}]
		\item[\textbf{4 -}]
	\end{itemize}
\end{document}