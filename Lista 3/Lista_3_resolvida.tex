\documentclass[12pt]{article}

\usepackage[utf8]{inputenc}
\usepackage[portuguese]{babel}
\usepackage{float}
\usepackage{graphicx}
\usepackage{amssymb}

\title{MAC0444 - Sistemas Baseados em Conhecimento \\
Lista de Exercícios No. 3
}
\author{Mateus Agostinho dos Anjos\\NUSP 9298191}
\date{\today}

\begin{document}
	\maketitle
	\begin{itemize}
		\item[\textbf{1 -}]
			\hfill\newline
			\begin{itemize}
				\item[\textbf{a) }]
					\hfill\newline
					O conceito abaixo significa:\\
					Um humano que não é do sexo feminino, casado(a) com um médico(a) e que,
					se tiver filhos(as), todos são médicos(as) ou professores(as). 
				\item[\textbf{b) }]
					\hfill\newline
					Considerando apenas as 4 pessoas mencionadas, João, Marta, Pedro
					e Olívia, \textbf{NÃO podemos afirmar que algum deles pertence a esse conceito.}\\
					Mesmo sabendo que todas as 4 pessoas citadas são humanas e diferentes entre si,
					que Jõao e Pedro não são do sexo feminino enquanto Marta e Olívia são do sexo feminino
					e que geriatra e cardiologista são profissões que caracterizam um médico(a),
					temos que:\\ 
					\textbf{João não pertence ao conceito}, pois não foi informado se ele é casado com
					um médico(a), \textbf{Marta e Olívia não pertencem ao conceito}, pois são do sexo
					feminino e \textbf{Pedro não pertence ao conceito}, pois tem Marta como filha e não
					sabemos se ela (Marta) é médica ou professora,
					portanto não sabemos se todos os filhos de Pedro são médicos(as) ou professores(as).
					\newpage
				\item[\textbf{c) }]
					\hfill\newline
					A resposta acima \textbf{mudaria apenas se Marta não existisse.}\\
					Note que, se Marta não existir, diríamos que Pedro pertence ao conceito, 
					pois é humano, não é do sexo feminino, é casado com uma médica (Olívia) e
					teria apenas 1 filho, João, que é médico (cardiologista), logo todos os filhos
					de Pedro são médicos(as) ou professores(as).
			\end{itemize}
		\item[\textbf{2 -}]
			\hfill\newline
			Vamos construir um contra-exemplo simples para mostrar que o
			axioma $Pessoa \sqcap \neg Homem \equiv Mulher$ não é
			consequência lógica de $\mathcal{T}$.
			\begin{center}
				$\Delta^\mathcal{I} = \lbrace \alpha, \beta, \gamma \rbrace$\\
				$Pessoa^\mathcal{I} = \lbrace \alpha, \beta, \gamma \rbrace$\\
				$Homem^\mathcal{I} = \lbrace \alpha \rbrace$\\
				$Mulher^\mathcal{I} = \lbrace \beta \rbrace$
			\end{center}
			Perceba que não contradizemos nada que está definido na $T$-Box $\mathcal{T}$,
			porém\\
			$\gamma: Pessoa \sqcap \neg Homem$ é verdadeiro.\\
			$\gamma: Mulher$ é falso.\\
			Logo concluímos que $Pessoa \sqcap \neg Homem \not\equiv Mulher$
			
		\item[\textbf{3 -}]
			\hfill\newline
			Traduzindo o axioma para uma sentença da Lógica de Primeira Ordem temos:\\
			\begin{footnotesize}
			$\forall x( PaiDeMedicos(x) \rightarrow \\ (\exists y(temFilho(x, y) \wedge (Homem(y) \vee Mulher(y))) \wedge (\forall z (temFilho(x, z) \rightarrow Medico(z)))))$
			\end{footnotesize}			
		\item[\textbf{4 -}]
			\hfill\newline
			Queremos mostrar que a base de conhecimento composta pelos conceitos (e equivalências) do
			enunciado tem como consequência:\\ $Vegano \sqsubseteq Vegetatiano$. \\
			Para isso mostraremos,
			utilizando tableaux, que a base de conhecimento acrescida da sentença $(Vegano \sqcap \neg 
			Vegetariano)(x)$ é inconsistente, ou
			seja, que em qualquer interpretação que satisfaça a base de conhecimento não existe nenhum $x$
			tal que $Vegano(x)$ seja verdadeiro e $Vegetariano(x)$ seja falso.\\
			\newpage
			Resolvendo o tableaux temos:
			\hfill\newline	\hfill\newline			
			\includegraphics[width=\linewidth]{tableaux_questao_4.png}
			\newpage	
			Fechamos todos os ramos do tableaux com uma contradição, por isso
			mostramos que não há interpretação que satisfaça a base de
			conhecimento do enunciado e que mapeie o conceito
			$Vegano \sqcap \neg Vegetariano$ para um conjunto não vazio,
			logo provamos que $Vegano \sqsubseteq Vegetatiano$ é
			consequência desta base de conhecimento.
	\end{itemize}
\end{document}