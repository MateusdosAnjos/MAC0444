\documentclass[12pt]{article}

\usepackage[utf8]{inputenc}
\usepackage[portuguese]{babel}
\usepackage{float}
\usepackage{graphicx}
\usepackage{amssymb}

\title{MAC0444 - Sistemas Baseados em Conhecimento \\
Lista de Exercícios No. 3
}
\author{Mateus Agostinho dos Anjos\\NUSP 9298191}
\date{\today}

\begin{document}
	\maketitle
	\begin{itemize}
		\item[\textbf{1 -}]
			\hfill\newline
			\begin{itemize}
				\item[\textbf{a) }]
					\hfill\newline
					O conceito abaixo significa:\\
					Um humano que não é do sexo feminino, casado(a) com um médico(a) e que tem filho(a) que
					é médico(a) ou professor(a). 
				\item[\textbf{b) }]
					\hfill\newline
					Primeiro devemos fazer algumas considerações:\\
					(Tais considerações são consideradas como senso comum, porém
					precisam ser especificadas)
					\subitem\textbf{•}
					Todos as 4 pessoas citadas são humanas e diferentes entre si.
					\subitem\textbf{•}
					Jõao e Pedro não são do sexo feminino enquanto Marta e Olívia são do sexo feminino.
					\subitem\textbf{•}
					Geriatra e cardiologista são profissões que caracterizam um médico(a).
					\newline
					A partir disso podemos dizer que \textbf{Pedro pertence a esse conceito}, pois é humano,
					não é do sexo feminino, é casado com Olívia que é uma médica e tem o filho João que é
					médico.
					\newpage
				\item[\textbf{c) }]
					\hfill\newline
					A resposta depende da pessoa excluída:
					\newline
					\subitem\textbf{•}
					Se excluirmos João:\\
					Não podemos concluir que Pedro tem um filho(a) que é médico(a) ou
					professor(a), portanto não podemos concluir que Pedro pertence ao conceito.\\
					Marta e Olívia são do sexo feminino, portanto não pertencem ao conceito.\\
					Desta forma, ao excluirmos João, não podemos afirmar que alguém pertence ao
					conceito.
					\newline
					\subitem\textbf{•}
					Se excluirmos Marta, a resposta do item b) não se altera, pois ela não tem
					nenhuma influência na questão.
					\newline
					\subitem\textbf{•}
					Se excluirmos Pedro:\\
					Não podemos concluir que João tem um filho(a) que é médico(a) ou
					professor(a) e nem que é casado com uma médica, portanto não podemos concluir que João
					pertence ao conceito.\\
					Marta e Olívia são do sexo feminino, portanto não pertencem ao conceito.\\
					Desta forma, ao excluirmos Pedro, não podemos afirmar que alguém pertence ao
					conceito.
					\newline
					\subitem\textbf{•}
					Se excluirmos Olívia:\\
					Não podemos concluir que Pedro é casado com uma médica, portanto não podemos concluir que 
					Pedro pertence ao conceito.\\
					Não podemos concluir que João tem um filho(a) que é médico(a) ou
					professor(a) e nem que é casado com uma médica, portanto não podemos concluir que João
					pertence ao conceito.\\
					Marta é do sexo feminino, portanto não pertence ao conceito.\\
					Desta forma, ao excluirmos Olívia, não podemos afirmar que alguém pertence ao
					conceito.
			\end{itemize}
		\newpage
		\item[\textbf{2 -}]
			\hfill\newline
			Pela T-Box $\mathcal{T}$ sabemos que toda valoração que satisfaz $Mulher$ também satisfaz
			$Pessoa$ ($Mulher \sqsubseteq Pessoa$), além disso sabemos também que toda valoração
			que satisfaz $Mulher$ também satisfaz $\neg Homem$ ($Mulher \sqsubseteq \neg Homem$),
			portanto toda valoração de $Mulher$ satisfaz tanto $Pessoa$ quanto $\neg Homem$, ou seja:\\ 
			$Mulher \sqsubseteq Pessoa \sqcap \neg Homem$.\\
			
			Além disso sabemos que toda valoração de $Homem$ satisfaz tanto $Pessoa$ 
			($Homem \sqsubseteq Pessoa$) quanto $\neg Mulher$ ($Homem \sqsubseteq \neg Mulher$),
			portanto, a partir de $\mathcal{T}$, $\neg Homem$ deve satisfazer $\neg Pessoa$ ou
			$Mulher$, sendo assim $Pessoa \sqcap \neg Homem$ só pode ser satisfeito se $Mulher$
			for satisfeito, ou seja:\\
			$Pessoa \sqcap \neg Homem \sqsubseteq Mulher$.\\
			
			Mostramos que: $Mulher \sqsubseteq Pessoa \sqcap \neg Homem$\\
			Também mostramos que: $Pessoa \sqcap \neg Homem \sqsubseteq Mulher$\\
			Então: $Pessoa \sqcap \neg Homem \equiv Mulher$.
		\item[\textbf{3 -}]
			\hfill\newline
			Traduzindo o axioma para uma sentença da Lógica de Primeira Ordem temos:\\
			\begin{footnotesize}
			$\forall x( PaiDeMedicos(x) \rightarrow \\ (\exists y(temFilho(x, y) \wedge (Homem(y) \vee Mulher(y))) \wedge (\forall z (temFilho(x, z) \rightarrow Medico(z)))))$
			\end{footnotesize}			
		\item[\textbf{4 -}]
	\end{itemize}
\end{document}