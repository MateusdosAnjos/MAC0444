\documentclass[12pt]{article}

\usepackage[utf8]{inputenc}
\usepackage[portuguese]{babel}
\usepackage{float}
\usepackage{tikz-qtree}
\usepackage{latexsym}

\title{MAC0444 - Sistemas Baseados em Conhecimento \\
Lista de Exercícios No. 2
}
\author{Mateus Agostinho dos Anjos\\NUSP 9298191}
\date{\today}

\begin{document}
	\maketitle
	\begin{itemize}
		\item[\textbf{1 -}]
			\hfill\newline
			Predicados:\\
			$fezEx(x)$ = x fez os exercícios\\
			$vaiBem(x)$ = x vai bem na prova\\
			$mediaAlta(x)$ = x fica com media alta\\
			$aprovado(x, y)$ = x é aprovado em y \\
			\newline
			Formalizando as sentenças do enunciado chegamos em:\\
			$$\forall x \ (fezEx(x) \rightarrow vaiBem(x))$$
			$$\forall x \ (vaiBem(x) \rightarrow mediaAlta(x))$$
			$$\forall x \ (mediaAlta(x) \rightarrow aprovado(x, mac444))$$
			$$fezEx(\textnormal{joão})$$
			$$vaiBem(\textnormal{maria})$$
			Base de conhecimento:\\
			$$[\neg fezEx(x), \ vaiBem(x)]$$
			$$[\neg vaiBem(x), \ mediaAlta(x)]$$
			$$[\neg mediaAlta(x), \ aprovado(x, mac444)] $$
			$$[fezEx(\textnormal{joão})]$$
			$$[vaiBem(\textnormal{maria})]$$
			$$[\neg aprovado(joao, mac444)]$$
			Veja que inserimos $[\neg aprovado(joao, mac444)]$ na base de conhecimento, pois
			é a negação do nosso objetivo. Sendo assim, se chegarmos na cláusula vazia com
			a partir desta base de conhecimento estará provado que $aprovado(joao, mac444)$
			é consequência lógica das sentenças do enunciado.\\
			\newline
			Utilizando a \textbf{resolução SLD} temos:\\
			\begin{center}
				$\neg aprovado(joao, mac444)$\\
				$\downarrow$ x/joão\\
				$\neg mediaAlta(\textnormal{joão})$\\
				$\downarrow$\\
				$\neg vaiBem(\textnormal{joão})$\\
				$\downarrow$\\
				$\neg fezEx(\textnormal{joão})$\\
				$\downarrow$\\
				$[ \ ]$
				
			\end{center}
	\end{itemize}
\end{document}