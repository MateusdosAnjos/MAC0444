\documentclass[12pt]{article}

\usepackage[utf8]{inputenc}
\usepackage[portuguese]{babel}
\usepackage{float}
\usepackage{tikz-qtree}
\usepackage{latexsym}

\title{MAC0444 - Sistemas Baseados em Conhecimento \\
Lista de Exercícios No. 2
}
\author{Mateus Agostinho dos Anjos\\NUSP 9298191}
\date{\today}

\begin{document}
	\maketitle
	\begin{itemize}
		\item[\textbf{1 -}]
			\hfill\newline
			Predicados:\\
			$fezEx(x)$ = x fez os exercícios\\
			$vaiBem(x)$ = x vai bem na prova\\
			$mediaAlta(x)$ = x fica com media alta\\
			$aprovado(x, y)$ = x é aprovado em y \\
			\newline
			Formalizando as sentenças do enunciado chegamos em:\\
			$$\forall x \ (fezEx(x) \rightarrow vaiBem(x))$$
			$$\forall x \ (vaiBem(x) \rightarrow mediaAlta(x))$$
			$$\forall x \ (mediaAlta(x) \rightarrow aprovado(x, mac444))$$
			$$fezEx(\textnormal{joão})$$
			$$vaiBem(\textnormal{maria})$$
			Base de conhecimento (KB):\\
			\begin{center}
				\begin{tabular}{c l}
				1. & $[\neg fezEx(x), \ vaiBem(x)]$\\
				2. & $[\neg vaiBem(x), \ mediaAlta(x)]$\\
				3. & $[\neg mediaAlta(x), \ aprovado(x, mac444)] $\\
				4. & $[fezEx(\textnormal{joão})]$\\
				5. & $[vaiBem(\textnormal{maria})]$\\
				6. & $[\neg aprovado(\textnormal{joão}, mac444)]$\\
				\end{tabular}
			\end{center}
			Veja que inserimos $[\neg aprovado(\textnormal{joão}, mac444)]$ na base de 
			conhecimento, pois	é a negação do nosso objetivo. Sendo assim, se chegarmos na 
			cláusula vazia com a partir desta base de conhecimento estará provado que 
			$aprovado(\textnormal{joão},	 mac444)$ é consequência lógica das sentenças do
			 enunciado.\\
			\newline
			Utilizando a \textbf{resolução SLD} temos:\\
			\begin{center}
				\begin{tabular}{c c}
					$\neg aprovado(\textnormal{joão}, mac444)$ & (resolve com 3. e x/joão)\\
					$\downarrow$ & \\
					$\neg mediaAlta(\textnormal{joão})$ & (resolve com 2.)\\
					$\downarrow$ & \\
					$\neg vaiBem(\textnormal{joão})$ & (resolve com 1.)\\
					$\downarrow$ & \\
					$\neg fezEx(\textnormal{joão})$ & (resolve com 4.)\\
					$\downarrow$ & \\
					$[ \ ]$ & \\
				\end{tabular}
			\end{center}
			Sendo assim provamos que: $KB \ \cup \  \{ \neg aprovado(\textnormal{joão}, 
			mac444) \}$ é insatisfazível, portanto $aprovado(\textnormal{joão}, mac444)$  
			é consequência lógica de nossa base de conhecimento.\\
	\end{itemize}
\end{document}