\documentclass[12pt]{article}

\usepackage[utf8]{inputenc}
\usepackage[portuguese]{babel}
\usepackage{float}
\usepackage{tikz-qtree}
\usepackage{latexsym}

\title{MAC0444 - Sistemas Baseados em Conhecimento \\
Lista de Exercícios No. 2
}
\author{Mateus Agostinho dos Anjos\\NUSP 9298191}
\date{\today}

\begin{document}
	\maketitle
	\begin{itemize}
		\item[\textbf{1 -}]
			\hfill\newline
			Predicados:\\
			$fezEx(x)$ = x fez os exercícios\\
			$vaiBem(x)$ = x vai bem na prova\\
			$mediaAlta(x)$ = x fica com media alta\\
			$aprovado(x, y)$ = x é aprovado em y \\
			\newline
			Formalizando as sentenças do enunciado chegamos em:\\
			$$\forall x \ (fezEx(x) \rightarrow vaiBem(x))$$
			$$\forall y \ (vaiBem(y) \rightarrow mediaAlta(y))$$
			$$\forall z \ (mediaAlta(z) \rightarrow aprovado(z, mac444))$$
			$$fezEx(\textnormal{João})$$
			$$vaiBem(\textnormal{Maria})$$
			Base de conhecimento (KB):\\
			\begin{center}
				\begin{tabular}{c l}
				1. & $[\neg fezEx(x), \ vaiBem(x)]$\\
				2. & $[\neg vaiBem(y), \ mediaAlta(y)]$\\
				3. & $[\neg mediaAlta(z), \ aprovado(z, mac444)] $\\
				4. & $[fezEx(\textnormal{João})]$\\
				5. & $[vaiBem(\textnormal{Maria})]$\\
				6. & $[\neg aprovado(\textnormal{João}, mac444)]$\\
				\end{tabular}
			\end{center}
			Veja que inserimos $[\neg aprovado(\textnormal{João}, mac444)]$ na base de 
			conhecimento, pois	é a negação do nosso objetivo. Sendo assim, se chegarmos na 
			cláusula vazia com a partir desta base de conhecimento estará provado que 
			$aprovado(\textnormal{João}, mac444)$ é consequência lógica das sentenças do
			 enunciado.\\
			\newline
			Utilizando a \textbf{resolução SLD} temos:\\
			\begin{center}
				\begin{tabular}{c c}
					$\neg aprovado(\textnormal{João}, mac444)$ & (resolve com 3. e z/João)\\
					$\downarrow$ & \\
					$\neg mediaAlta(\textnormal{João})$ & (resolve com 2. e y/João)\\
					$\downarrow$ & \\
					$\neg vaiBem(\textnormal{João})$ & (resolve com 1. e x/João)\\
					$\downarrow$ & \\
					$\neg fezEx(\textnormal{João})$ & (resolve com 4.)\\
					$\downarrow$ & \\
					$[ \ ]$ & \\
				\end{tabular}
			\end{center}
			Sendo assim provamos que: $KB \ \cup \  \{ \neg aprovado(\textnormal{João}, 
			mac444) \}$ é insatisfazível, portanto $aprovado(\textnormal{João}, mac444)$  
			é consequência lógica de nossa base de conhecimento.\\ \\
			A \textbf{resolução SLD} será semelhante para Maria, portanto temos:\\
			Base de conhecimento (KB):\\
			\begin{center}
				\begin{tabular}{c l}
				1. & $[\neg fezEx(x), \ vaiBem(x)]$\\
				2. & $[\neg vaiBem(y), \ mediaAlta(y)]$\\
				3. & $[\neg mediaAlta(z), \ aprovado(z, mac444)] $\\
				4. & $[fezEx(\textnormal{João})]$\\
				5. & $[vaiBem(\textnormal{Maria})]$\\
				6. & $[\neg aprovado(\textnormal{Maria}, mac444)]$\\
				\end{tabular}
			\end{center}
			Utilizando a \textbf{resolução SLD} temos:\\
			\begin{center}
				\begin{tabular}{c c}
					$\neg aprovado(\textnormal{Maria}, mac444)$ & (resolve com 3. e z/Maria)\\
					$\downarrow$ & \\
					$\neg mediaAlta(\textnormal{Maria})$ & (resolve com 2. e y/Maria)\\
					$\downarrow$ & \\
					$\neg vaiBem(\textnormal{Maria})$ & (resolve com 5.)\\
					$\downarrow$ & \\
					$[ \ ]$ & \\
				\end{tabular}
			\end{center}
		\item[\textbf{2 -}]
			\hfill\newline
			Temos a Base de Conhecimento (KB) reescrita com variáveis renomeadas para evitar confusões
			na resolução do exercício:
			\begin{center}
				\begin{tabular}{c l}
				1. & $[\neg A_1(x), \neg A_2(x), P(x)]$\\
				2. & $[\neg B_1(y), \neg B_2(y), A_1(y)]$\\
				3. & $[\neg B_3(z), \neg B4(z), A_2(z)]$\\
				4. & $[B_1(a)]$\\
				5. & $[B_2(a)]$\\
				6. & $[B_3(a)]$\\
				7. & $[B_4(a)]$\\
				\end{tabular}
			\end{center}
			\begin{itemize}	
				\item[\textbf{a) }]					
					\hfill\newline
					Para mostrar o passo a passo do procedimento de encadeamento para
					trás (backward chaining) devemos começar identificando as implicações
					da Base de Conhecimento.\\
					Seguindo a ordem acima temos:\\ 
					(note que utilizamos $\leftarrow$ nas implicações)\\
					\begin{center}
						\begin{tabular}{c l}
							1. & $\forall x (P(x) \leftarrow A_1(x) \land A_2(x))$\\
							2. & $\forall y (A_1(y) \leftarrow B_1(y) \land B_2(y))$\\
							3. & $\forall z (A_2(z) \leftarrow B_3(z) \land B_4(z))$\\
							4. & $B_1(a)$\\
							5. & $B_2(a)$\\
							6. & $B_3(a)$\\
							7. & $B_4(a)$\\
						\end{tabular}
					\end{center}
					A partir destas implicações o passo a passo pode ser mostrado a partir da figura
					abaixo, sendo que cada passo gera 1 nível da árvore.
				\item[\textbf{b) }]
			\end{itemize}	
	\end{itemize}
\end{document}